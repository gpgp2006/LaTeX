\documentclass{article}
\usepackage{graphicx}
\usepackage{amsmath}
\usepackage{pgfplots}
\usepackage[margin=1.5cm]{geometry}
\usepackage{adjustbox}
\usepackage{hyperref}
\usepackage{cite}
\usepackage{listings}
\usepackage{xcolor}
\usepackage{comment}
\usepackage{pdfpages}
\usepackage{enumitem}
\usepackage{pgfplotstable}
\usepackage{booktabs}
\usepackage{changepage}
\usepackage[utf8]{inputenc}
\usepackage[portuguese]{babel}

\lstset{ 
  literate= {á}{{\'a}}1 
            {é}{{\'e}}1 
            {í}{{\'i}}1 
            {ó}{{\'o}}1 
            {ú}{{\'u}}1
} 

\hypersetup{
  colorlinks=true,
  linkcolor=blue,
  urlcolor=blue,
}

\pgfplotsset{compat=1.17}

\definecolor{codegreen}{rgb}{0,0.6,0}
\definecolor{codegray}{rgb}{0.5,0.5,0.5}
\definecolor{codepurple}{rgb}{0.58,0,0.82}
\definecolor{backcolour}{rgb}{0.95,0.95,0.92}

\lstdefinestyle{mystyle}{
    backgroundcolor=\color{backcolour},   
    commentstyle=\color{codegreen},
    keywordstyle=\color{magenta},
    numberstyle=\tiny\color{codegray},
    stringstyle=\color{codepurple},
    basicstyle=\ttfamily\footnotesize,
    breakatwhitespace=false,         
    breaklines=true,                 
    captionpos=b,                    
    keepspaces=true,                 
    numbers=left,                    
    numbersep=5pt,                   
    showspaces=false,                
    showstringspaces=false,
    showtabs=false,                  
    tabsize=2
}

\lstset{style=mystyle}

\title{Atividade UML - draw.io}
\author{Gabriel Gaspar, Isaac Fischer, Kimberly Rotman e Yasmin Brancaleone}
\date{}

\begin{document}

\maketitle

\begin{itemize}[leftmargin=*, itemsep=0.4em]
  \item \textbf{Nome da ferramenta:} draw.io

  \item \textbf{Diagramas suportados:} Fluxograma, diagrama de processo BPMN, fluxo de dados, UML, diagrama entidade-relacionamento, diagrama de rede, diagrama de circuito eletrônico, mapas mentais, organograma, arquitetura de software.

  \item \textbf{Endereço web de acesso:} \href{https://draw.io}{draw.io}

  \item \textbf{Engenharia reversa:} Não é suportado pelo draw.io.

  \item \textbf{Geração de código:} Não é suportado pelo draw.io.

  \item \textbf{Geração de documentação:} Não é suportado pelo draw.io.

  \item \textbf{Notação da UML:} É suportado parcialmente pelo draw.io. Como ele é basicamente uma ferramenta de desenho, o draw.io suporta a notação visual totalmente, mas o mesmo não valida ou interpreta os elementos UML. O draw.io não identifica se alguma regra da UML foi violada.

  \item \textbf{Formato de arquivo gerado pela ferramenta:} O draw.io é capaz de exportar em PNG, JPG, SVG, WEBP, PDF, HTML e XML.

  \item \textbf{Capaz de exportar diagramas que podem ser usados em outras ferramentas:} Sim, desde que a ferramenta de destino suporte XML, mas não segue o padrão XMI da UML.

  \item \textbf{Versão testada:} v26.2.15

  \item \textbf{Evolução da ferramenta e data da última atualização da ferramenta:} A evolução dela é possível de ser acompanhada no \href{https://github.com/jgraph/drawio}{GitHub} deles. A data da última atualização da ferramenta foi no dia 26 de abril de 2025.

  \item \textbf{Usuários ou instituições que usam essa ferramenta:} Uma boa parte dos alunos do 3° semestre do curso de Bacharelado em Ciência da Computação, no Instituto Federal do Paraná, utiliza o draw.io para modelagem de banco de dados. Algumas empresas também utilizam a plataforma, como American Express, Atlassian Confluence e UPS.

  \item \textbf{Licença:} O código presente no GitHub está sob licença Apache v2. Os ícones utilizados são licenciados sob a CC BY 4.0.

  \item \textbf{Facilidade de uso:} É uma ferramenta de extrema facilidade de uso, com uma interface simples e poderosa e controles fáceis de se usar.

  \item \textbf{Avaliação pessoal de pontos fortes e pontos fracos:}  
  Alguns pontos fortes são: o draw.io é gratuito e open source, tem uma interface intuitiva, está localizado em português, não exige instalação, tem integração com serviços em nuvem como Google Drive, OneDrive e Dropbox, suporta diversos tipos de diagramas, colaboração em tempo real, exportação em vários formatos e uma vasta biblioteca de templates, além de ter curva de aprendizado baixa.  
  Pontos fracos: não suporta engenharia reversa, não gera código a partir de UML, não usa o padrão XMI, depende de navegador para colaboração e não tem recursos avançados.
\end{itemize}

\end{document}
