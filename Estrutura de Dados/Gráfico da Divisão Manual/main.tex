\documentclass{article}
\usepackage{graphicx} % Required for inserting images
\usepackage{pgfplots} % Required for inserting graphs
\usepackage[margin=1.5cm]{geometry} % Required for creating margin
\usepackage[portuguese]{babel}
\pgfplotsset{compat=1.18}

\title{Gráfico de comparação entre o tempo de divisão manual e a divisão com barra, em µs}
\author{Gabriel de Paula Gaspar Pinto}
\date{}

\begin{document}
\maketitle
\begin{center}
\begin{tikzpicture}
  \begin{axis}[
    title={Comparação entre o tempo de divisão manual e a divisão com "/", em µs},
    xlabel={Quantidade de vezes de execução do programa},
    ylabel={Tempo para divisão (µs)},
    grid=major,
    width = 15cm,
    height = 10cm,
    legend pos=north east
  ]
    \addplot [color=red, mark=*] coordinates {
      (1, 1662) (2, 1255) (3, 1027) (4, 1056) (5, 1136) (6, 1185) (7, 1028) (8, 1029) (9, 1013) (10, 1028) (11, 1028) (12, 2450) (13, 1015) (14, 1029) (15, 1023) (16, 1020) (17, 2328) (18, 1011) (19, 1013) (20, 1013) (21, 1014) (22, 1008) (23, 1008) (24, 1072) (25, 2276) (26, 1568) (27, 1130) (28, 1006) (29, 1007) (30, 2047) (31, 2340) (32, 2072) (33, 1011) (34, 1097) (35, 1094) (36, 1035) (37, 1065) (38, 1059) (39, 1042) (40, 994) (41, 983) (42, 1557) (43, 1028) (44, 998) (45, 986) (46, 968) (47, 995) (48, 983) (49, 1246) (50, 1011) (51, 1051) (52, 1001) (53, 1007) (54, 1020) (55, 1008) (56, 1243) (57, 1215) (58, 1025) (59, 1006) (60, 1008)
    };

    \addplot [color=blue, mark=square*] coordinates {
    (1, 0) (2, 0) (3, 0) (4, 0) (5, 0) (6, 0) (7, 0) (8, 0) (9, 0) (10, 0) (11, 0) (12, 0) (13, 0) (14, 0) (15, 0) (16, 0) (17, 0) (18, 0) (19, 0) (20, 0) (21, 0) (22, 0) (23, 0) (24, 0) (25, 0) (26, 0) (27, 0) (28, 0) (29, 0) (30, 0) (31, 0) (32, 0) (33, 0) (34, 0) (35, 0) (36, 0) (37, 0) (38, 0) (39, 0) (40, 0) (41, 0) (42, 0) (43, 0) (44, 0) (45, 0) (46, 0) (47, 0) (48, 0) (49, 0) (50, 0) (51, 0) (52, 0) (53, 0) (54, 0) (55, 0) (56, 0) (57, 0) (58, 0) (59, 0) (60, 0)
    };
    \addlegendentry{Divisão manual}
    \addlegendentry{Divisão com /}
  \end{axis}
\end{tikzpicture}
\end{center}

\end{document}