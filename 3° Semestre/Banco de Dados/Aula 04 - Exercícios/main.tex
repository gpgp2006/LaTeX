\documentclass{article}
\usepackage{graphicx}
\usepackage{amsmath}
\usepackage{pgfplots}
\usepackage[margin=1.5cm]{geometry}
\usepackage{adjustbox}
\usepackage{hyperref}
\usepackage{cite}
\usepackage{listings}
\usepackage{xcolor}
\usepackage{comment}
\usepackage{graphicx}
\usepackage{pdfpages}
\usepackage{enumitem}
\usepackage{pgfplotstable}
\usepackage{booktabs}
\usepackage{adjustbox}
\usepackage{changepage}
\usepackage[utf8]{inputenc}
\lstset{ 
literate= {á}{{\'a}}1 
    {é}{{\'e}}1 
    {í}{{\'i}}1 
    {ó}{{\'o}}1 
    {ú}{{\'u}}1
} 

\hypersetup{
colorlinks=true,
linkcolor=blue,
urlcolor=blue,
}

\pgfplotsset{compat=1.17}
\usepackage[portuguese]{babel}

\definecolor{codegreen}{rgb}{0,0.6,0}
\definecolor{codegray}{rgb}{0.5,0.5,0.5}
\definecolor{codepurple}{rgb}{0.58,0,0.82}
\definecolor{backcolour}{rgb}{0.95,0.95,0.92}

\lstdefinestyle{mystyle}{
    backgroundcolor=\color{backcolour},   
    commentstyle=\color{codegreen},
    keywordstyle=\color{magenta},
    numberstyle=\tiny\color{codegray},
    stringstyle=\color{codepurple},
    basicstyle=\ttfamily\footnotesize,
    breakatwhitespace=false,         
    breaklines=true,                 
    captionpos=b,                    
    keepspaces=true,                 
    numbers=left,                    
    numbersep=5pt,                  
    showspaces=false,                
    showstringspaces=false,
    showtabs=false,                  
    tabsize=2
}

\lstset{style=mystyle}

\title{Consultas SQL - Aula 04}
\author{Gabriel de Paula Gaspar Pinto}
\date{}

\begin{document}

\maketitle

\section*{Exercício 3}

\begin{enumerate}[label=\alph*.]
    \item De acordo com o que a Mariana quer, esta query está errada, porque não retorna qual garota vendeu mais biscoitos, mas retorna todas as vendas de cada garota, ordenadas pelo nome, alfabeticamente. A query retorna a tabela a seguir:
    
    \begin{center}
    \pgfplotstabletypeset[
        col sep=comma,
        string type,
        columns={nome,valor_venda},
        columns/nome/.style={column name=Nome},
        columns/valor_venda/.style={
            column name=Valor da Venda,
            column type={r},
            dec sep align
        },
        every head row/.style={before row=\toprule, after row=\midrule},
        every last row/.style={after row=\bottomrule}
    ]{data.csv}
    \end{center}

    \item 
    \begin{enumerate}[label=\roman*., labelsep=0.5em, leftmargin=*]
        
        \item 
        \begin{adjustwidth}{1em}{0pt}
            \begin{lstlisting}[language=SQL]
                SELECT nome, SUM(valor_venda)
                FROM garota, vendas_biscoito
                WHERE garota.id_garota = vendas_biscoito.id_garota
                GROUP BY garota.id_garota
            \end{lstlisting}
        \end{adjustwidth}

        \item
        \begin{adjustwidth}{1em}{0pt}
            \begin{lstlisting}[language=SQL]
                SELECT nome, MIN(valor_venda), MAX(valor_venda)
                FROM garota, vendas_biscoito
                WHERE garota.id_garota = vendas_biscoito.id_garota
                GROUP BY garota.id_garota;
            \end{lstlisting}
        \end{adjustwidth}

        \item 
        \begin{adjustwidth}{1em}{0pt}
            \begin{lstlisting}[language=SQL]
                SELECT nome, COUNT(DISTINCT data_venda) AS quantidade_vendas
                FROM garota, vendas_biscoito
                WHERE garota.id_garota = vendas_biscoito.id_garota
                GROUP BY garota.id_garota
            \end{lstlisting}
        \end{adjustwidth}
        
        \item 
        \begin{adjustwidth}{1em}{0pt}
            \begin{lstlisting}[language=SQL]
                SELECT nome, SUM(valor_venda) AS total
                FROM garota, vendas_biscoito
                WHERE garota.id_garota = vendas_biscoito.id_garota
                GROUP BY garota.id_garota
                ORDER BY total DESC
                LIMIT 2;
            \end{lstlisting}
        \end{adjustwidth}

        \item 
        \begin{adjustwidth}{1em}{0pt}
            \begin{lstlisting}[language=SQL]
                SELECT nome, SUM(valor_venda) AS total
                FROM garota, vendas_biscoito
                WHERE garota.id_garota = vendas_biscoito.id_garota
                GROUP BY garota.id_garota
                ORDER BY total DESC
                LIMIT 1 OFFSET 1;
            \end{lstlisting}
        \end{adjustwidth}

        \item 
        \begin{adjustwidth}{1em}{0pt}
            \begin{lstlisting}[language=SQL]
                SELECT nome, email
                FROM garota
                WHERE email IS NOT NULL
            \end{lstlisting}
        \end{adjustwidth}

        \item 
        \begin{adjustwidth}{1em}{0pt}
            \begin{lstlisting}[language=SQL]
                SELECT nome, telefone
                FROM garota
                WHERE email IS NULL
            \end{lstlisting}
        \end{adjustwidth}

        \item 
        \begin{adjustwidth}{1em}{0pt}
            \begin{lstlisting}[language=SQL]
                SELECT nome, email
                FROM garota
                WHERE email LIKE '%gmail.com'
            \end{lstlisting}
        \end{adjustwidth}

        \item 
        \begin{adjustwidth}{1em}{0pt}
            \begin{lstlisting}[language=SQL]
                SELECT nome, telefone
                FROM garota
                WHERE telefone LIKE '41%'
            \end{lstlisting}
        \end{adjustwidth}

        \item 
        \begin{adjustwidth}{1em}{0pt}
            \begin{lstlisting}[language=SQL]
                SELECT nome, nascimento
                FROM garota
                WHERE nome IN ('Ana','Júlia');
            \end{lstlisting}
        \end{adjustwidth}

        \item 
        \begin{adjustwidth}{1em}{0pt}
            \begin{lstlisting}[language=SQL]
                SELECT id_venda, data_venda
                FROM vendas_biscoito
                WHERE data_venda BETWEEN '2024-06-05' AND '2024-06-08'
            \end{lstlisting}
        \end{adjustwidth}

        \item 
        \begin{adjustwidth}{1em}{0pt}
            \begin{lstlisting}[language=SQL]
                SELECT nome AS Nome, 
                SUBSTRING_INDEX(endereco, ',', 1) AS 'Endereco (Logradouro)', 
                SUBSTRING_INDEX(endereco, ',', -1) AS 'Endereco (Número)'
                FROM garota                
            \end{lstlisting}
        \end{adjustwidth}

        \item 
        \begin{adjustwidth}{1em}{0pt}
            \begin{lstlisting}[language=SQL]
                SELECT ROUND(AVG(valor_venda), 2) AS 'Valor médio'
                FROM vendas_biscoito
            \end{lstlisting}
        \end{adjustwidth}

        \item 
        \begin{adjustwidth}{1em}{0pt}
            \begin{lstlisting}[language=SQL]
                SELECT nome, ROUND(AVG(valor_venda), 2) AS 'Valor médio'
                FROM garota, vendas_biscoito
                WHERE garota.id_garota = vendas_biscoito.id_garota
                GROUP BY garota.id_garota
            \end{lstlisting}
        \end{adjustwidth}

        \item 
        \begin{adjustwidth}{1em}{0pt}
            \begin{lstlisting}[language=SQL]
                SELECT nome AS 'Nome', ROUND(AVG(valor_venda), 2) AS 'Valor médio'
                FROM garota, vendas_biscoito
                WHERE garota.id_garota = vendas_biscoito.id_garota
                GROUP BY garota.id_garota
                HAVING AVG(valor_venda) > 14
            \end{lstlisting}
        \end{adjustwidth}

        \item 
        \begin{adjustwidth}{1em}{0pt}
            \begin{lstlisting}[language=SQL]
                SELECT nome AS Nome, ROUND(AVG(valor_venda), 2) AS 'Média individual'
                FROM garota, vendas_biscoito
                WHERE garota.id_garota = vendas_biscoito.id_garota
                GROUP BY garota.id_garota
                HAVING AVG(valor_venda) > (SELECT AVG(valor_venda) FROM vendas_biscoito)                
            \end{lstlisting}
        \end{adjustwidth}

        \item 
        \begin{adjustwidth}{1em}{0pt}
            \begin{lstlisting}[language=SQL]
                SELECT nome AS Nome, 
                DATE_FORMAT(nascimento, '%d/%m/%Y') AS 'Data de Nascimento'
                FROM garota
            \end{lstlisting}
        \end{adjustwidth}

        \item 
        \begin{adjustwidth}{1em}{0pt}
            \begin{lstlisting}[language=SQL]
                SELECT nome AS Nome, TIMESTAMPDIFF(YEAR, nascimento, CURDATE()) as 'Idade'
                FROM garota
            \end{lstlisting}
        \end{adjustwidth}

        \item 
        \begin{adjustwidth}{1em}{0pt}
            \begin{lstlisting}[language=SQL]
                SELECT nome
                FROM garota
                WHERE id_garota NOT IN (SELECT id_garota FROM vendas_biscoito)
            \end{lstlisting}
        \end{adjustwidth}

    \end{enumerate}

\end{enumerate}

\section*{Exercício 4}

\begin{enumerate}[label=\alph*.]

    \item Para Mariana poder calcular a pontuação, de acordo com a fórmula dada, é necessário usar a seguinte query:
    
    \begin{adjustwidth}{1em}{0pt}
        \begin{lstlisting}[language=SQL]
            SELECT nome AS Nome, 
            TIMESTAMPDIFF(YEAR, nascimento, CURDATE()) AS Idade, 
            ROUND(AVG(valor_venda), 2) AS 'Média de venda', 
            ROUND(AVG(valor_venda) - (TIMESTAMPDIFF(YEAR, nascimento, CURDATE()) * 0.5), 2) AS 'Pontuacao'
            FROM garota, vendas_biscoito
            WHERE garota.id_garota = vendas_biscoito.id_garota
            GROUP BY garota.id_garota
            ORDER BY 4 DESC
        \end{lstlisting}
    \end{adjustwidth}

\end{enumerate}


\section*{Considerações Finais}
\paragraph{} O comando SQL do exercício 12 da questão 3 está com a grafia incorreta devido à um problema com o pacote \textit{listings} do \LaTeX{}, no qual não estava compilando quando a palavra "endereço" estava grafada corretamente, assim como no exercício 4, no qual não compila se a palavra "pontuação" esteja grafada e acentuada corretamente. Este e outros trabalhos feitos usando \LaTeX{} estão disponíveis em meu \href{https://github.com/gpgp2006/LaTeX}{GitHub}, junto com os arquivos utilizados.



\end{document}