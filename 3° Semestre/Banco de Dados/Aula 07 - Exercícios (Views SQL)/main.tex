\documentclass{article}
\usepackage{graphicx}
\usepackage{amsmath}
\usepackage{pgfplots}
\usepackage[margin=1.5cm]{geometry}
\usepackage{adjustbox}
\usepackage{hyperref}
\usepackage{cite}
\usepackage{listings}
\usepackage{xcolor}
\usepackage{comment}
\usepackage{graphicx}
\usepackage{pdfpages}
\usepackage{enumitem}
\usepackage{pgfplotstable}
\usepackage{booktabs}
\usepackage{adjustbox}
\usepackage{changepage}
\usepackage[utf8]{inputenc}
\lstset{ 
literate= {á}{{\'a}}1 
    {é}{{\'e}}1 
    {í}{{\'i}}1 
    {ó}{{\'o}}1 
    {ú}{{\'u}}1
} 

\hypersetup{
colorlinks=true,
linkcolor=blue,
urlcolor=blue,
}

\pgfplotsset{compat=1.17}
\usepackage[portuguese]{babel}

\definecolor{codegreen}{rgb}{0,0.6,0}
\definecolor{codegray}{rgb}{0.5,0.5,0.5}
\definecolor{codepurple}{rgb}{0.58,0,0.82}
\definecolor{backcolour}{rgb}{0.95,0.95,0.92}

\lstdefinestyle{mystyle}{
    backgroundcolor=\color{backcolour},   
    commentstyle=\color{codegreen},
    keywordstyle=\color{magenta},
    numberstyle=\tiny\color{codegray},
    stringstyle=\color{codepurple},
    basicstyle=\ttfamily\footnotesize,
    breakatwhitespace=false,         
    breaklines=true,                 
    captionpos=b,                    
    keepspaces=true,                 
    numbers=left,                    
    numbersep=5pt,                  
    showspaces=false,                
    showstringspaces=false,
    showtabs=false,                  
    tabsize=2
}

\lstset{style=mystyle}

\title{SQL Views - Aula 07}
\author{Gabriel de Paula Gaspar Pinto}
\date{}

\begin{document}

\maketitle

\section*{Exercício 1}

\begin{enumerate}[label=\alph*)]

    % A)
    \item 
    \begin{lstlisting}[language=SQL]
        CREATE VIEW disciplinas_geral AS
        SELECT
        d.nome, a.nome
        FROM
        disciplina AS d
        LEFT JOIN
        aluno_cursa_disciplina AS c
        ON d.codigo_disciplina = c.codigo_disciplina
        LEFT JOIN
        aluno AS a
        ON c.matricula_aluno = a.matricula_aluno
    \end{lstlisting}

    % B)
    \item
    \begin{lstlisting}[language=SQL]
        CREATE VIEW alunos_geral AS
        SELECT
        a.nome AS nome_aluno,
        d.nome AS nome_disciplina
        FROM
        aluno AS a
        LEFT JOIN
        aluno_cursa_disciplina AS c
        USING (matricula_aluno)
        LEFT JOIN
        disciplina AS d
        USING (codigo_disciplina)
    \end{lstlisting}

    % C)
    \item 
    \begin{lstlisting}[language=SQL]
        CREATE VIEW disciplinas_alunos_geral AS
        SELECT nome_disciplina, nome_aluno
        FROM
        disciplinas_geral
        
        UNION
        
        SELECT nome_disciplina, nome_aluno
        FROM
        alunos_geral        
    \end{lstlisting}

    % D)
    \item 
    \begin{lstlisting}[language=SQL]
        SELECT nome_disciplina
        FROM disciplinas_alunos_geral
        WHERE nome_aluno IS NULL
    \end{lstlisting}

    % E)
    \item 
    \begin{lstlisting}[language=SQL]
        SELECT nome_aluno
        FROM disciplinas_alunos_geral
        WHERE nome_disciplina IS NULL
    \end{lstlisting}

    % F
    \item 
    \begin{lstlisting}[language=SQL]
        SELECT *
        FROM disciplinas_alunos_geral
        WHERE !(nome_disciplina IS NULL OR nome_aluno IS NULL)        
    \end{lstlisting}

\end{enumerate}

\section*{Exercício 3}
Usando a seguinte query para criar uma view para controle de inserção de dados:

    \begin{lstlisting}[language=SQL]
        CREATE VIEW insercao_aluno AS
        SELECT *
        FROM aluno
        WHERE (sexo = 'M') OR (sexo = 'F')
        WITH CHECK OPTION;
    \end{lstlisting}

Usando a seguinte query, para colocar dados no banco de dados:

    \begin{lstlisting}[language=SQL]
        INSERT INTO insercao_aluno (cpf, nome, telefone, endereco_cep, endereco_num, sexo)
        VALUES ('12345678912', 'Sabrina Carpenter', '41988341294', '11111111', 42, 'X')
    \end{lstlisting}

O MySQL retorna o seguinte erro: Error code: 1369. CHECK OPTION failed 'mydb.insercao\_aluno'. Por fim, corrigindo a query e trocando o sexo de 'X' para 'F', é possível inserir os valores corretamente.

\section*{Exercício 4}
Para alterar a view 'insercao\_aluno', foi usada a seguinte query:

    \begin{lstlisting}[language=SQL]
        CREATE OR REPLACE VIEW insercao_aluno AS
        SELECT *
        FROM aluno
        WHERE (((sexo = 'M') OR (sexo = 'F')) AND 
        (cpf REGEXP '^\[0-9]{3}.[0-9]{3}.[0-9]{3}-[0-9]{2}') AND 
        (endereco_cep REGEXP '^\[0-9]{5}-[0-9]{3}') AND
        (telefone REGEXP '^\[(][0-9]{2}[)][0-9]{5}-[0-9]{4}'))
        WITH CHECK OPTION
    \end{lstlisting}

\section*{Considerações Finais}
\paragraph{}Este e outros trabalhos que feitos por mim, usando \LaTeX{} estão disponíveis em meu \href{https://github.com/gpgp2006/LaTeX}{GitHub}, junto com os arquivos utilizados.

\end{document}