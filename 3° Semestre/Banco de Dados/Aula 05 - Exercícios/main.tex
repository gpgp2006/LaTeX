\documentclass{article}
\usepackage{graphicx}
\usepackage{amsmath}
\usepackage{pgfplots}
\usepackage[margin=1.5cm]{geometry}
\usepackage{adjustbox}
\usepackage{hyperref}
\usepackage{cite}
\usepackage{listings}
\usepackage{xcolor}
\usepackage{comment}
\usepackage{graphicx}
\usepackage{pdfpages}
\usepackage{enumitem}
\usepackage{pgfplotstable}
\usepackage{booktabs}
\usepackage{adjustbox}
\usepackage{changepage}
\usepackage[utf8]{inputenc}
\lstset{ 
literate= {á}{{\'a}}1 
    {é}{{\'e}}1 
    {í}{{\'i}}1 
    {ó}{{\'o}}1 
    {ú}{{\'u}}1
} 

\hypersetup{
colorlinks=true,
linkcolor=blue,
urlcolor=blue,
}

\pgfplotsset{compat=1.17}
\usepackage[portuguese]{babel}

\definecolor{codegreen}{rgb}{0,0.6,0}
\definecolor{codegray}{rgb}{0.5,0.5,0.5}
\definecolor{codepurple}{rgb}{0.58,0,0.82}
\definecolor{backcolour}{rgb}{0.95,0.95,0.92}

\lstdefinestyle{mystyle}{
    backgroundcolor=\color{backcolour},   
    commentstyle=\color{codegreen},
    keywordstyle=\color{magenta},
    numberstyle=\tiny\color{codegray},
    stringstyle=\color{codepurple},
    basicstyle=\ttfamily\footnotesize,
    breakatwhitespace=false,         
    breaklines=true,                 
    captionpos=b,                    
    keepspaces=true,                 
    numbers=left,                    
    numbersep=5pt,                  
    showspaces=false,                
    showstringspaces=false,
    showtabs=false,                  
    tabsize=2
}

\lstset{style=mystyle}

\title{SQL Joins - Aula 06}
\author{Gabriel de Paula Gaspar Pinto}
\date{}

\begin{document}

\maketitle

\section*{Exercício 3}

\begin{enumerate}[label=\alph*)]

    \item A query não é satisfatória, já que não mostra os alunos e as matérias em que foram matriculados, mas sim a combinação de todas as linhas, múltiplas vezes. A query corrigida, fica:

    % A)
    \begin{lstlisting}[language=SQL]
        SELECT
        a.nome, c.codigo_disciplina
        FROM
        aluno AS a
        INNER JOIN
        aluno_cursa_disciplina AS c
        ON a.matricula_aluno = c.matricula_aluno
    \end{lstlisting}

    É possível omitir a linha 7 e trocar de "INNER JOIN" para "NATURAL JOIN", usando o MySQL, que o resultado será o mesmo.

    % B)
    \item 
    \begin{lstlisting}[language=SQL]
        SELECT
        a.nome, c.codigo_disciplina
        FROM
        aluno AS a
        LEFT JOIN
        aluno_cursa_disciplina AS c
        ON a.matricula_aluno = c.matricula_aluno
        UNION
        SELECT
        a.nome, c.codigo_disciplina
        FROM
        aluno AS a
        RIGHT JOIN
        aluno_cursa_disciplina AS c
        ON a.matricula_aluno = c.matricula_aluno
    \end{lstlisting}

    % C)
    \item
    \begin{lstlisting}[language=SQL]
        SELECT
        a.nome, c.codigo_disciplina
        FROM
        aluno AS a
        LEFT JOIN
        aluno_cursa_disciplina AS c
        ON a.matricula_aluno = c.matricula_aluno
        WHERE c.codigo_disciplina IS NULL
    \end{lstlisting}

    % D)
    \item Como as tabelas de alunos e de disciplinas não são conectadas diretamente, é necessário usar dois LEFT JOINS para mesclar as informações das duas tabelas, através da tabela de aluno\_cursa\_disciplina.
    \begin{lstlisting}[language=SQL]
        SELECT
        d.nome, a.nome
        FROM
        disciplina AS d
        LEFT JOIN
        aluno_cursa_disciplina AS c
        ON d.codigo_disciplina = c.codigo_disciplina
        LEFT JOIN
        aluno AS a
        ON c.matricula_aluno = a.matricula_aluno
    \end{lstlisting}

    % E)
    \item Mostra o nome da disciplina quando nenhum aluno está matriculado. Se um aluno está matriculado, a disciplina está presente em C, caso contrário, a disciplina não está presente em C. Com isso, mostra todas as disciplinas presentes em D (tabela com TODAS as disciplinas) que não tenham seus correspondentes em C (tabela com alunos e disciplinas).
    \begin{lstlisting}[language=SQL]
        SELECT
        d.nome
        FROM
        disciplina AS d
        LEFT JOIN
        aluno_cursa_disciplina AS c
        ON c.codigo_disciplina = d.codigo_disciplina
        WHERE c.codigo_disciplina IS NULL
    \end{lstlisting}

    % F)
    \item Para mostrar todas os alunos e suas respectivas disciplinas, é necessário usar o LEFT JOIN, da tabela de alunos para aluno\_cursa\_disciplina, que conterá todos os alunos, mesmo que não tenha nenhuma disciplina atrelado ao mesmo. Com o USING, a coluna de matricula\_aluno é usada para unir as tabelas, já que esta coluna está presente nas duas tabelas.
    \begin{lstlisting}[language=SQL]
        SELECT a.nome, a.matricula_aluno, c.codigo_disciplina
        FROM
        aluno AS a
        LEFT JOIN
        aluno_cursa_disciplina AS c
        USING (matricula_aluno)        
    \end{lstlisting}

    % G)
    \item Para mostrar o nome de todas as disciplinas e o nome de seus respectivos alunos, usa-se o LEFT JOIN da tabela de disciplinas para aluno\_cursa\_disciplina, com o USING (codigo\_disciplina), mostrando todas as disciplinas, sem exceção. Com o segundo LEFT JOIN, junta-se a tabela aluno\_cursa\_disciplina com a de alunos. Por fim, com o último USING, somente os alunos que estão matriculados serão mostrados, junto com todas as disciplinas. 
    \begin{lstlisting}[language=SQL]
        SELECT d.nome, a.nome
        FROM
        disciplina AS d
        LEFT JOIN
        aluno_cursa_disciplina AS c
        USING (codigo_disciplina)
        LEFT JOIN
        aluno AS a
        USING (matricula_aluno)
    \end{lstlisting}

    % H)
    \item Fazendo um FULL JOIN, acontece a mesma coisa que aconteceu na questão anterior, com a exceção que todas as disciplinas (com alunos ou sem) e todos os alunos (com disciplinas ou não) são mostrados.
    \begin{lstlisting}[language=SQL]
        SELECT d.nome, a.nome
        FROM
        disciplina AS d
        LEFT JOIN
        aluno_cursa_disciplina AS c
        USING (codigo_disciplina)
        LEFT JOIN
        aluno AS a
        USING (matricula_aluno)
        
        UNION
        
        SELECT d.nome, a.nome
        FROM
        disciplina AS d
        RIGHT JOIN
        aluno_cursa_disciplina AS c
        USING (codigo_disciplina)
        RIGHT JOIN
        aluno AS a
        USING (matricula_aluno)
    \end{lstlisting}


\end{enumerate}

\section*{Exercício 4}

\begin{enumerate}[label=\alph*)]

    % A)
    \item 
    \begin{lstlisting}[language=SQL]
        SELECT *
        FROM 
        a
        INNER JOIN
        b
        ON a.pk = b.pk       
    \end{lstlisting}

    % B)
    \item 
    \begin{lstlisting}[language=SQL]
        SELECT *
        FROM
        a
        LEFT JOIN
        b
        ON a.pk = b.pk
    \end{lstlisting}

    % C)
    \item 
    \begin{lstlisting}[language=SQL]
        SELECT *
        FROM
        a
        RIGHT JOIN
        b
        ON a.pk = b.pk
    \end{lstlisting}

    % D)
    \item 
    \begin{lstlisting}[language=SQL]
        SELECT *
        FROM
        a
        LEFT JOIN
        b
        ON a.pk = b.pk
        WHERE b.pk IS NULL
    \end{lstlisting}

    % E)
    \item 
    \begin{lstlisting}[language=SQL]
        SELECT *
        FROM
        a
        RIGHT JOIN
        b
        ON b.pk = a.pk
        WHERE a.pk IS NULL
    \end{lstlisting}

    % F)
    \item 
    \begin{lstlisting}[language=SQL]
        SELECT *
        FROM
        a
        LEFT JOIN
        b
        ON a.pk = b.pk
        
        UNION

        SELECT *
        FROM
        a
        RIGHT JOIN
        b
        ON a.pk = b.pk
    \end{lstlisting}

    % G)
    \item 
    \begin{lstlisting}[language=SQL]
        SELECT *
        FROM
        a
        LEFT JOIN
        b
        ON a.pk = b.pk
        WHERE b.pk IS NULL

        UNION

        SELECT *
        FROM
        a
        RIGHT JOIN
        b
        ON b.pk = a.pk
        WHERE a.pk IS NULL
    \end{lstlisting}

\end{enumerate}

\section*{Considerações finais}
\paragraph{}A partir da questão D), eu coloquei explicações de como cada query funciona porque eu estava tendo dificuldades. Explicar como elas funcionam me ajudou a entender e a fixar o conteúdo. Este e outros trabalhos feitos usando \LaTeX{} estão disponíveis em meu \href{https://github.com/gpgp2006/LaTeX}{GitHub}, com seus códigos-fonte. 


\end{document}