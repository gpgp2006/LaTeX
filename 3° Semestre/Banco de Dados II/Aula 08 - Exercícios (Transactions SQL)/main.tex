\documentclass{article}
\usepackage{graphicx}
\usepackage{amsmath}
\usepackage{pgfplots}
\usepackage[margin=1.5cm]{geometry}
\usepackage{adjustbox}
\usepackage{hyperref}
\usepackage{cite}
\usepackage{listings}
\usepackage{xcolor}
\usepackage{comment}
\usepackage{graphicx}
\usepackage{pdfpages}
\usepackage{enumitem}
\usepackage{pgfplotstable}
\usepackage{booktabs}
\usepackage{adjustbox}
\usepackage{changepage}
\usepackage[utf8]{inputenc}
\lstset{ 
literate= {á}{{\'a}}1 
    {é}{{\'e}}1 
    {í}{{\'i}}1 
    {ó}{{\'o}}1 
    {ú}{{\'u}}1
} 

\hypersetup{
colorlinks=true,
linkcolor=blue,
urlcolor=blue,
}

\pgfplotsset{compat=1.17}
\usepackage[portuguese]{babel}

\definecolor{codegreen}{rgb}{0,0.6,0}
\definecolor{codegray}{rgb}{0.5,0.5,0.5}
\definecolor{codepurple}{rgb}{0.58,0,0.82}
\definecolor{backcolour}{rgb}{0.95,0.95,0.92}

\lstdefinestyle{mystyle}{
    backgroundcolor=\color{backcolour},   
    commentstyle=\color{codegreen},
    keywordstyle=\color{magenta},
    numberstyle=\tiny\color{codegray},
    stringstyle=\color{codepurple},
    basicstyle=\ttfamily\footnotesize,
    breakatwhitespace=false,         
    breaklines=true,                 
    captionpos=b,                    
    keepspaces=true,                 
    numbers=left,                    
    numbersep=5pt,                  
    showspaces=false,                
    showstringspaces=false,
    showtabs=false,                  
    tabsize=2
}

\lstset{style=mystyle}

\title{SQL Transactions - Aula 08}
\author{Gabriel de Paula Gaspar Pinto}
\date{}

\begin{document}

\maketitle

\section*{Exercício 1}

    \paragraph{} Transação, em um contexto de banco de dados, é um conjunto de comandos SQL individuais que são agrupados e executados juntos de maneira sequencial. Logo, ou tudo acontece ou nada acontece. Por exemplo, caso haja um erro no banco de dados, a luz caia ou tenha um problema deste tipo, no meio de uma transação, nenhum comando será guardado e o banco voltará estar do jeito que estava antes desta transação. Mas, caso não aconteça nenhum problema durante a transação, ela é concluída e suas alterações são efetivadas no banco.
    \paragraph{} Um exemplo de transações da vida real, é um banco (do setor financeiro). Quando se vai sacar dinheiro, a parte do banco de dados faz um SELECT para mostrar a quantidade de dinheiro na conta e um UPDATE para subtrair o valor sacado e logo após o dinheiro sai do caixa eletrônico. Caso haja um problema entre o processo de atualizar o valor da conta e o saque do dinheiro, se não fosse usado transações, o cliente simplesmente perderia o dinheiro, mas como as transações são usadas, o UPDATE é revertido para o valor anterior, como se nada tivesse acontecido.
    
\section*{Exercício 2}

\begin{enumerate}[label=\alph*)]

    %atomicidade
    \item Em um sistema bancário, ao tentar transferir R\$100,00 de uma conta A para B. O dinheiro pode ser removido da conta A mas não ser creditado na conta B, resultando no dinheiro sumindo.

    %consistência
    \item Em uma conta poupança, o saldo nunca pode ser negativo, mas sem consistência, é possível que o saldo seja atualizado para um número negativo, resultando em problemas.

    %isolamento
    \item Em um sistema sem isolamento, de um show, por exemplo, quando restar somente 1 ingresso, duas pessoas, simultaneamente, podem ver que há somente 1 ingresso restante. As duas conseguem comprar, resultando em um estoque negativo.

    %durabilidade
    \item Um pedido pode ser finalizado e confirmado ao cliente, mas antes das mudanças serem gravadas, ocorre uma queda de energia, resultando em um pedido perdido, mesmo após a confirmação.

\end{enumerate}

\section*{Exercício 3}

\begin{enumerate}[label=\alph*)]

    %a
    \item Para inserir o usuário, um simples INSERT INTO foi suficiente, mas para inserir os conteúdos da conta corrente e da conta poupança, foi necessário usar LAST\_INSERT\_ID(), que retorna a última chave primária inserida, já que a chave primária do usuário é auto incremental, o que torna a chave primária desconhecida, já que não é algo imutável, como o CPF de uma pessoa. Em relação à transações, tudo ocorreu de acordo com o esperado, sem nenhum erro.
    
    %b
    \item Como esperado, o rollback fez com que o banco de dados retornasse ao estado anterior ao início da transação.
    \newline
    \newline
    Isso também aconteceria em situações como ao inserir dados em várias tabelas e um dos inserts falhar, quando uma condição de integridade for violada durante uma atualização em lote ou quando há bloqueios ou deadlocks durante a transação são exemplos que podem ser dados. 
    \newline
    \newline
    O rollback é útil para proteger o banco de dados em caso de falha humana. Por exemplo, se alguma pessoa remover dados sensíveis, sem querer, ainda é possível de usar o rollback para que o sistema consiga recuperar os dados apagados por erro. Outro caso em que o rollback é útil, é em falhas não humanas, como por exemplo, quedas de energia. Se o banco de dados ficar sem energia no meio de uma transação, a transação não é concluída e o banco retorna ao estado anterior ao início da transação.
    
    %c
    \item Após iniciar a transação na primeira conexão e subtrair R\$500,00 da conta poupança, mas sem rodar o commit, a segunda conexão só conseguia rodar selects na tabela, se rodasse um update, a segunda aba perdia conexão com o banco de dados. Mas, após dar o commit na primeira transação, foi possível remover mais R\$400,00 da conta, mesmo que ela só tenha R\$300,00, restando em um saldo de R\$100,00 negativos em uma conta poupança. Rodando o update da segunda conexão e, antes de 30 segundos dar commit na primeira transação, o erro de desconexão é não acontece. Isso acontece porque o padrão de nível de isolamento do MySQL é "repeatable read". Para previnir o erro de ter uma conta poupança negativa, é possível atualizar a query de update, para algo como mostrado abaixo ou simplesmente mudar o nível de isolamento do banco para "serializable", que não permite conexões concorrentes.
    
    \begin{lstlisting}[language=SQL]
        UPDATE conta_poupanca SET saldo = saldo - 400 WHERE id_conta = 1 AND saldo >= 400;
    \end{lstlisting}

    %d
    \item Após iniciar a transação na primeira conexão, não é possível atualizar nenhum dado na segunda conexão até a transação da primeira conexão ser finalizada. Após 30 segundos, caso a transação da primeira conexão não tenha sido finalizada, a segunda conexão perde a conexão com o banco de dados.
    
\end{enumerate}

\section*{Exercício 4}

\begin{enumerate}[label=\alph*)]

    %a
    \item
    \begin{lstlisting}[language=SQL]
        SET @num1 = 20;
        SET @num2 = 30;
        SET @num3 = @num1 + @num2;
        SELECT @num3;
    \end{lstlisting}

    %b
    \item Com a seguinte query, é possível fazer o pedido:
    \begin{lstlisting}[language=SQL]
        SET @a = 5;
        SET @b = 2;
        SET @aux = 0;

        SET @aux = @a;
        SET @a = @b;
        SET @b = @aux;

        SELECT @a AS a, @b AS b;
    \end{lstlisting}

\end{enumerate}

\section*{Exercício 5}

\begin{enumerate}[label=\alph*)]

    %a
    \item 
    \begin{lstlisting}[language=SQL]
        UPDATE conta_corrente SET saldo = 400 WHERE id_conta = 1;
        UPDATE conta_poupanca SET saldo = 300 WHERE id_conta = 1;
    \end{lstlisting}

    %b
    \item 
    \begin{lstlisting}[language=SQL]
        SET @transferencia = 300;
    \end{lstlisting}

    %c
    \item 
    \begin{lstlisting}[language=SQL]
        UPDATE conta_corrente SET saldo = saldo - @transferencia WHERE id_conta = 1;
        UPDATE conta_poupanca SET saldo = saldo + @transferencia WHERE id_conta = 1;
    \end{lstlisting}

    %d
    \item As contas estavam com os valores exatos.
\end{enumerate}

\section*{Questões de concurso}

\begin{enumerate}
    %1
    \item a) Certo
    
    %2
    \item a) As transações controlam melhor apenas a concorrência.
    
    %3
    \item a) Certo
    
    %4
    \item A questão não possui enunciado completo, mas considerando o que foi dado, está incorreto.
    
    %5
    \item c) uma transação lê dados gravados por outra transação que ainda não foi confirmada (committed).
    
    %6
    \item c) A atomicidade é a propriedade que assegura que as atualizações relacionadas e dependentes ocorram dentro dos limites da transação ou nenhuma atualização será efetivada no banco de dados.
    
    %7
    \item b) Errado
    
    %8
    \item d) II e III.
    
    %9
    \item c) Propriedade de durabilidade, que garante que, após uma transação ser concluída com êxito, as alterações feitas no banco de dados persistem, mesmo se houver falhas do sistema.
    
    %10
    \item c) Isolação e Atomicidade.
    
    %11
    \item e) Isolamento
    
    %12
    \item b) mesmo na ocorrência de falhas no sistema de banco de dados, após o término da transação.
    
    %13
    \item d) Atomicidade, Consistência, Durabilidade e Isolamento.
    
    %14
    \item c) durabilidade.
    
    %15
    \item d) O isolamento resolve os efeitos decorrentes da execução de transações concorrentes, em que cada transação é executada de forma que as operações parciais das demais transações não afetem a transação atual.
    
    %16
    \item e) transações.
    
    %17
    \item a) Certo
    
    %18
    \item b) Errado
    
    %19
    \item b) Errado

\end{enumerate}

\end{document}