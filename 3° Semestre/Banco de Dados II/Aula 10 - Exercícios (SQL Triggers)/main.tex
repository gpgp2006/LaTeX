\documentclass{article}
\usepackage{graphicx}
\usepackage{amsmath}
\usepackage{pgfplots}
\usepackage{array}
\usepackage[margin=1.5cm]{geometry}
\usepackage{adjustbox}
\usepackage{hyperref}
\usepackage{cite}
\usepackage{listings}
\usepackage{xcolor}
\usepackage{comment}
\usepackage{graphicx}
\usepackage{pdfpages}
\usepackage{enumitem}
\usepackage{pgfplotstable}
\usepackage{booktabs}
\usepackage{adjustbox}
\usepackage{changepage}
\usepackage[utf8]{inputenc}
\lstset{ 
literate= {á}{{\'a}}1 
    {é}{{\'e}}1 
    {í}{{\'i}}1 
    {ó}{{\'o}}1 
    {ú}{{\'u}}1
} 

\hypersetup{
colorlinks=true,
linkcolor=blue,
urlcolor=blue,
}

\pgfplotsset{compat=1.17}
\usepackage[portuguese]{babel}

\definecolor{codegreen}{rgb}{0,0.6,0}
\definecolor{codegray}{rgb}{0.5,0.5,0.5}
\definecolor{codepurple}{rgb}{0.58,0,0.82}
\definecolor{backcolour}{rgb}{0.95,0.95,0.92}

\lstdefinestyle{mystyle}{
    backgroundcolor=\color{backcolour},   
    commentstyle=\color{codegreen},
    keywordstyle=\color{magenta},
    numberstyle=\tiny\color{codegray},
    stringstyle=\color{codepurple},
    basicstyle=\ttfamily\footnotesize,
    breakatwhitespace=false,         
    breaklines=true,                 
    captionpos=b,                    
    keepspaces=true,                 
    numbers=left,                    
    numbersep=5pt,                  
    showspaces=false,                
    showstringspaces=false,
    showtabs=false,                  
    tabsize=2
}

\lstset{style=mystyle}

\title{SQL Triggers - Aula 10}
\author{Gabriel de Paula Gaspar Pinto}
\date{}

\begin{document}

\maketitle

\section*{Exercício 1}

\begin{enumerate}[label=\alph*)]

    %letra A
    \item Gatilhos (triggers) são um mecanismo que permite executar automaticamente um bloco de comandos SQL quando uma determinada ação ocorre sobre uma tabela, como a inserção, atualização ou exclusão de dados.
    
    \item As instruções que podem disparar um gatilho são o INSERT, UPDATE e DELETE. Essas ações podem ser associadas a um gatilho configurado para ser executado antes (BEFORE) ou depois (AFTER) da operação ser realizada.
    
    \item Durante a execução de um gatilho, o SGDB disponibiliza duas tabelas temporárias, o OLD e o NEW. As duas armazenam os valores das linhas afetadas. A tabela OLD representa os dados antes da modificação, enquanto a tabela NEW representa os dados após a modificação.
    \par No caso do INSERT, apenas a tabela NEW é disponibilizada, já que a linha inserida não existia antes. Em um DELETE, somente a tabela OLD estará disponível, já que a linha será removida e não haverá novos dados. Por fim, em um UPDATE, tanto quando a tabela NEW e a tabela OLD estão disponíveis, já que uma linha não será removida ou inserida, mas somente editada, permitindo que os valores anteriores e os novos sejam comparados.

\end{enumerate}

\section*{Exercício 2}

\noindent
\begin{minipage}[t]{0.23\textwidth}
\centering
\textbf{tabela1} \\
\begin{tabular}{|c|}
\hline
a1 \\
\hline
1 \\
3 \\
1 \\
7 \\
1 \\
8 \\
4 \\
4 \\
\hline
\end{tabular}
\end{minipage}
\hfill
\begin{minipage}[t]{0.23\textwidth}
\centering
\textbf{tabela2} \\
\begin{tabular}{|c|}
\hline
a2 \\
\hline
1 \\
3 \\
1 \\
7 \\
1 \\
8 \\
4 \\
4 \\
\hline
\end{tabular}
\end{minipage}
\hfill
\begin{minipage}[t]{0.23\textwidth}
\centering
\textbf{tabela3} \\
\begin{tabular}{|c|}
\hline
a3 \\
\hline
2 \\
5 \\
6 \\
9 \\
10 \\
\hline
\end{tabular}
\end{minipage}
\hfill
\begin{minipage}[t]{0.23\textwidth}
\centering
\textbf{tabela4} \\
\begin{tabular}{|c|c|}
\hline
a4 & b4 \\
\hline
1 & 3 \\
2 & 0 \\
3 & 1 \\
4 & 2 \\
5 & 0 \\
6 & 0 \\
7 & 1 \\
8 & 1 \\
9 & 0 \\
10 & 0 \\
\hline
\end{tabular}
\end{minipage}

\section*{Exercício 3}

\begin{lstlisting}[language=SQL]
    DELIMITER $$

    CREATE TRIGGER atualizar_categoria
    BEFORE UPDATE ON cliente
    FOR EACH ROW
    BEGIN
    IF NEW.total_gasto < 1000 THEN
        SET NEW.categoria = 'bronze';
    ELSEIF NEW.total_gasto < 5000 THEN
        SET NEW.categoria = 'prata';
    ELSE
        SET NEW.categoria = 'ouro';
    END IF;
    END $$

    DELIMITER ;
\end{lstlisting}

\section*{Exercício 4}

\begin{enumerate}[label=\alph*)]
    
    %letra A
    \item  
    \begin{lstlisting}[language=SQL]
        DELIMITER $$

        CREATE TRIGGER reduzir_quantidade_disponivel
        AFTER INSERT ON emprestimo
        FOR EACH ROW
        BEGIN
        UPDATE livro SET quantidade_disponivel = quantidade_disponivel - 1 
        WHERE isbn = NEW.livro_isbn;
        END $$

        DELIMITER ;
    \end{lstlisting}

    %letra B
    \item 
    \begin{lstlisting}[language=SQL]
        DELIMITER $$

        CREATE TRIGGER aumentar_quantidade_disponivel
        AFTER UPDATE ON emprestimo
        FOR EACH ROW
        BEGIN
        IF OLD.ativo = '1' AND NEW.ativo = '0' THEN
            UPDATE livro SET quantidade_disponivel = quantidade_disponivel + 1 
            WHERE isbn = NEW.livro_isbn;
        END IF;
        END $$

        DELIMITER ;
    \end{lstlisting}

    %letra C
    \item 
    \begin{lstlisting}[language=SQL]
        CREATE VIEW notificacoes_disponiveis AS
        SELECT 
        s.nome AS nome_socio,
        st.numero AS telefone,
        l.titulo AS livro
        FROM notificacao n
        JOIN socio s ON s.cpf = n.socio_cpf
        JOIN socio_telefone st ON st.socio_cpf = s.cpf
        JOIN livro l ON l.isbn = n.livro_isbn
        WHERE l.quantidade_disponivel > 0;
    \end{lstlisting}

    %letra D
    \item 
    \begin{lstlisting}[language=SQL]
        CREATE VIEW emprestimos_atrasados AS
        SELECT s.nome AS nome_socio, l.titulo AS nome_livro, e.retirada, e.devolucao
        FROM emprestimo e
        JOIN socio s ON s.cpf = e.socio_cpf
        JOIN livro l ON l.isbn = e.livro_isbn
        WHERE e.ativo = '1' AND e.devolucao < CURDATE();
    \end{lstlisting}

\end{enumerate}

\section*{Exercício 5}

\begin{itemize}

    % insert
    \item INSERT
    \begin{lstlisting}[language=SQL]
        DELIMITER $$

        CREATE TRIGGER log_leilao_insert
        AFTER INSERT ON leilao
        FOR EACH ROW
        BEGIN
        INSERT INTO leilao_log (id_leilao, id_usuario, id_item, acao, hora, lance, usuario_bd) 
        VALUES (NEW.id_leilao, NEW.id_usuario, NEW.id_item, 'INSERT', NOW(), NEW.lance, CURRENT_USER());
        END$$

        DELIMITER ;

    \end{lstlisting}

    % update
    \item UPDATE
    \begin{lstlisting}[language=SQL]
        DELIMITER $$

        CREATE TRIGGER log_leilao_update
        AFTER UPDATE ON leilao
        FOR EACH ROW
        BEGIN
        INSERT INTO leilao_log (id_leilao, id_usuario, id_item, acao, hora, lance, usuario_bd) 
        VALUES (NEW.id_leilao, NEW.id_usuario, NEW.id_item, 'UPDATE', NOW(), NEW.lance, CURRENT_USER());
        END$$

        DELIMITER ;
    \end{lstlisting}

    % delete
    \item DELETE
    \begin{lstlisting}[language=SQL]
        DELIMITER $$

        CREATE TRIGGER log_leilao_delete
        AFTER DELETE ON leilao
        FOR EACH ROW
        BEGIN
        INSERT INTO leilao_log (id_leilao, id_usuario, id_item, acao, hora, lance, usuario_bd) 
        VALUES (OLD.id_leilao, OLD.id_usuario, OLD.id_item, 'DELETE', NOW(), OLD.lance, CURRENT_USER());
        END$$

        DELIMITER ;
    \end{lstlisting}

\end{itemize}

\end{document}