\documentclass{article}
\usepackage{graphicx}
\usepackage{amsmath}
\usepackage{pgfplots}
\usepackage[margin=1.5cm]{geometry}
\usepackage{adjustbox}
\usepackage{hyperref}
\usepackage{cite}
\usepackage{listings}
\usepackage{xcolor}
\usepackage{comment}
\usepackage{graphicx}
\usepackage{pdfpages}
\usepackage{enumitem}
\usepackage{pgfplotstable}
\usepackage{booktabs}
\usepackage{adjustbox}
\usepackage{changepage}
\usepackage[utf8]{inputenc}
\lstset{ 
literate= {á}{{\'a}}1 
    {é}{{\'e}}1 
    {í}{{\'i}}1 
    {ó}{{\'o}}1 
    {ú}{{\'u}}1
} 

\hypersetup{
colorlinks=true,
linkcolor=blue,
urlcolor=blue,
}

\pgfplotsset{compat=1.17}
\usepackage[portuguese]{babel}

\definecolor{codegreen}{rgb}{0,0.6,0}
\definecolor{codegray}{rgb}{0.5,0.5,0.5}
\definecolor{codepurple}{rgb}{0.58,0,0.82}
\definecolor{backcolour}{rgb}{0.95,0.95,0.92}

\lstdefinestyle{mystyle}{
    backgroundcolor=\color{backcolour},   
    commentstyle=\color{codegreen},
    keywordstyle=\color{magenta},
    numberstyle=\tiny\color{codegray},
    stringstyle=\color{codepurple},
    basicstyle=\ttfamily\footnotesize,
    breakatwhitespace=false,         
    breaklines=true,                 
    captionpos=b,                    
    keepspaces=true,                 
    numbers=left,                    
    numbersep=5pt,                  
    showspaces=false,                
    showstringspaces=false,
    showtabs=false,                  
    tabsize=2
}

\lstset{style=mystyle}

\title{Stored Procedures (SQL/PL) - Aula 09}
\author{Gabriel de Paula Gaspar Pinto}
\date{}

\begin{document}

\maketitle

\section*{Exercício 1}
    \paragraph{} Stored procedures, no SQL, é um segmento de código SQL armazenado no próprio banco de dados e pode ser chamado por um programa, pelo banco de dados ou até mesmo por outra stored procedure.

\section*{Exercício 2}
    \paragraph{} Os modos possíveis de passagem de parâmetros são o IN, OUT e INOUT. Um exemplo de IN, é quando um gerente de um banco quiser ver as transações de um cliente específico, o CPF de tal cliente será passado como IN. Um exemplo de OUT, é a pesquisa do salário de um funcionário, no qual a procedure retorna o salário por um parâmetro OUT. Por fim, um exemplo de INOUT, é um sistema de caixa de mercado, que calcula o preço de um item e aplica taxas e descontos (prática comum nos Estados Unidos), no qual o preço de um produto na prateleira é passado como IN e o preço com taxas/impostos e descontos retorna como OUT. 

\section*{Exercício 3}
    \begin{enumerate}[label=\alph*)]

        % letra A
        \item Para criar a procedure, foi usado o seguinte código:
        \begin{lstlisting}[language=SQL]
            DELIMITER //
                CREATE PROCEDURE somar(IN num1 INT, IN num2 INT, OUT soma INT)
                BEGIN
                    SET resultado = num1 + num2
                END
            DELIMITER ;
        \end{lstlisting}

        Agora, para usar a procedure feita, foi usado o seguinte código:
        \begin{lstlisting}[language=SQL]
            CALL somar(5, 7, @soma);
            SELECT @soma AS 'Soma'
        \end{lstlisting}

        % letra B
        \item Para criar a procedure, foi usado o seguinte código:
        \begin{lstlisting}[language=SQL]
            DELIMITER //
                CREATE PROCEDURE par_ou_impar(IN num INT, OUT resultado VARCHAR(5))
                BEGIN
                    IF MOD(num, 2) = 0 THEN
                        SET resultado = 'Par';
                    ELSE
                        SET resultado = 'Impar';
                    END IF;
                END //
            DELIMITER ;
        \end{lstlisting}

        Agora, para usar esta procedure, foi usado o seguinte código:
        \begin{lstlisting}[language=SQL]
            CALL par_ou_impar(58769, @resultado);
            SELECT @resultado AS 'Par ou Impar'
        \end{lstlisting}

    \end{enumerate}

\section*{Exercício 4}
    \begin{enumerate}[label=\alph*)]

        %letra A
        \item Para criar a função, foi utilizado o código abaixo. O MySQL estava reclamando da ausência do "DETERMINISTIC", então eu o coloquei.
        \begin{lstlisting}[language=SQL]
            DELIMITER //
            CREATE FUNCTION idade_garota (nasc DATE) RETURNS INT DETERMINISTIC
            BEGIN
                DECLARE idade INT;
                SET idade = TIMESTAMPDIFF(YEAR, nasc, CURDATE());
                RETURN idade;
            END //
            DELIMITER ;
        \end{lstlisting}

        Para retornar a idade de todas as garotas, foi usado a seguinte query:
        \begin{lstlisting}[language=SQL]
            SELECT nome, idade_garota(nascimento) AS idade FROM garota;
        \end{lstlisting}

        Mas, para retornar a idade de uma garota específica, usando o nome como parâmetro, a query fica:
        \begin{lstlisting}[language=SQL]
            SELECT idade_garota(nascimento) AS idade FROM garota WHERE nome = 'Ana'
        \end{lstlisting}

        %letra B
        \item Para criar a stored procedure da pontuação (Lista 4), foi usada a seguinte query:
        \begin{lstlisting}[language=SQL]
            DELIMITER //
            CREATE PROCEDURE pontuacao (IN id_nome VARCHAR(50), OUT idade INT, OUT media DECIMAL(10, 2), OUT total DECIMAL(10,2))
            BEGIN
                SELECT TIMESTAMPDIFF(YEAR, nascimento, CURDATE()) INTO idade FROM garota WHERE nome = id_nome; 
                SELECT ROUND(AVG(valor_venda), 2) INTO media FROM garota, vendas_biscoito WHERE garota.id_garota = vendas_biscoito.id_garota AND garota.nome = id_nome;
                SET total = ROUND(media - (idade * 0.5), 2);
            END //
            DELIMITER ;
        \end{lstlisting}

        Para chamar a procedure, usa-se a seguinte query, com o nome como parâmetro, para facilitar o uso da procedure.
        \begin{lstlisting}[language=SQL]
            CALL pontuacao('Beatriz', @idade, @media, @total);
            SELECT @idade AS 'Idade', @media AS 'Media', @total AS 'Pontuacao Total'; 
        \end{lstlisting}
        

    \end{enumerate}

\section*{Exercício 5}
\begin{enumerate}[label=\alph*)]

    %letra A
    \item Para criação da função de raiz quadrada, foi usada a seguinte query:
    \begin{lstlisting}[language=SQL]
        DELIMITER $$

        CREATE FUNCTION RaizQuadradaAproximada(c FLOAT)
        RETURNS FLOAT DETERMINISTIC
        BEGIN
            DECLARE x FLOAT;
            DECLARE funcao FLOAT;
            DECLARE erro FLOAT;

            SET x = c / 2.0;
            SET erro = ABS(c - (x * x));

            WHILE erro > 0.0001 DO
                SET funcao = 0.5 * (x + (c / x));
                SET x = funcao;
                SET erro = ABS(c - (x * x));
            END WHILE;

            RETURN x;
        END$$

        DELIMITER ;
    \end{lstlisting}

    Para retornar os valores, se usa a seguinte query, com "c" sendo a constante que você quer calcular a sua raiz quadrada:
    
    \begin{lstlisting}[language=SQL]
        SELECT RaizQuadradaAproximada(c);
    \end{lstlisting}
    
    %letra B
    \item A criação da função que calcula a fórmula de Bhaskara, se deu pela seguinte query

    \begin{lstlisting}[language=SQL]
        DELIMITER $$

        CREATE FUNCTION bhaskara(a FLOAT, b FLOAT, c FLOAT)
        RETURNS VARCHAR(200) DETERMINISTIC
        BEGIN
            DECLARE delta FLOAT;
            DECLARE raiz FLOAT;
            DECLARE x1 FLOAT;
            DECLARE x2 FLOAT;
            DECLARE resultado VARCHAR(200);

            IF a = 0 THEN
                RETURN 'Impossível calcular';
            END IF;

            SET delta = b * b - 4 * a * c;

            IF delta < 0 THEN
                RETURN 'Impossível calcular';
            END IF;

            SET raiz = SQRT(delta);
            SET x1 = (-b + raiz) / (2 * a);
            SET x2 = (-b - raiz) / (2 * a);

            SET resultado = CONCAT('R1 = ', ROUND(x1, 5), ', R2 = ', ROUND(x2, 5));

            RETURN resultado;
        END$$

        DELIMITER ;
    \end{lstlisting}

    Para retornar, usa-se a seguinte query, com a, b e c sendo as mesmas variáveis da fórmula de Bhaskara:

    \begin{lstlisting}[language=SQL]
        SELECT bhaskara(a, b, c);
    \end{lstlisting}

    %letra C
    \item Para criar a função de números romanos, foi usado a seguinte query:
    \begin{lstlisting}[language=SQL] 
        DELIMITER $$

        CREATE FUNCTION romano(arabico INT)
        RETURNS VARCHAR(100) DETERMINISTIC
        BEGIN
            DECLARE resultado VARCHAR(100) DEFAULT '';
            
            IF arabico < 1 OR arabico > 3000 THEN
                RETURN 'Numero menor que 1 ou maior que 3000';
            END IF;

            WHILE arabico >= 1000 DO
                SET resultado = CONCAT(resultado, 'M');
                SET arabico = arabico - 1000;
            END WHILE;

            IF arabico >= 900 THEN
                SET resultado = CONCAT(resultado, 'CM');
                SET arabico = arabico - 900;
            END IF;

            IF arabico >= 500 THEN
                SET resultado = CONCAT(resultado, 'D');
                SET arabico = arabico - 500;
            END IF;

            IF arabico >= 400 THEN
                SET resultado = CONCAT(resultado, 'CD');
                SET arabico = arabico - 400;
            END IF;

            WHILE arabico >= 100 DO
                SET resultado = CONCAT(resultado, 'C');
                SET arabico = arabico - 100;
            END WHILE;

            IF arabico >= 90 THEN
                SET resultado = CONCAT(resultado, 'XC');
                SET arabico = arabico - 90;
            END IF;

            IF arabico >= 50 THEN
                SET resultado = CONCAT(resultado, 'L');
                SET arabico = arabico - 50;
            END IF;

            IF arabico >= 40 THEN
                SET resultado = CONCAT(resultado, 'XL');
                SET arabico = arabico - 40;
            END IF;

            WHILE arabico >= 10 DO
                SET resultado = CONCAT(resultado, 'X');
                SET arabico = arabico - 10;
            END WHILE;

            IF arabico = 9 THEN
                SET resultado = CONCAT(resultado, 'IX');
                SET arabico = arabico - 9;
            END IF;

            IF arabico >= 5 THEN
                SET resultado = CONCAT(resultado, 'V');
                SET arabico = arabico - 5;
            END IF;

            IF arabico = 4 THEN
                SET resultado = CONCAT(resultado, 'IV');
                SET arabico = arabico - 4;
            END IF;

            WHILE arabico >= 1 DO
                SET resultado = CONCAT(resultado, 'I');
                SET arabico = arabico - 1;
            END WHILE;

            RETURN resultado;
        END$$

        DELIMITER ;
    \end{lstlisting}

    Para retornar um valor, usa-se a seguinte query, com arabico sendo o valor que você quer converter de arábico para romano:

    \begin{lstlisting}[language=SQL]
        SELECT romano(arabico);
    \end{lstlisting}

\end{enumerate}

\section*{Exercício 6}

\begin{enumerate}[label=\alph*)]
    %letra A
    \item A criação da query:
    \begin{lstlisting}[language=SQL]
        DELIMITER $$

        CREATE PROCEDURE detalha_usuario(IN in_id INT)
        BEGIN
        SELECT *
        FROM usuarios
        WHERE id_conta = in_id;
        END$$

        DELIMITER ;
    \end{lstlisting}

    Para retornar a procedure
    \begin{lstlisting}[language=SQL]
        CALL detalha_usuario(id);
    \end{lstlisting}

    %letra B
    \item A criação da query:
    \begin{lstlisting}[language=SQL]
        DELIMITER $$

        CREATE PROCEDURE saldo_corrente(IN in_id INT, OUT out_saldo DECIMAL(10,2))
        BEGIN
        SELECT IFNULL(SUM(saldo),0)
        INTO out_saldo
        FROM conta_corrente
        WHERE id_conta = in_id;
        END$$

        DELIMITER ;
    \end{lstlisting}

    Para retornar a procedure

    \begin{lstlisting}[language=SQL]
        SET @s = 0;
        CALL saldo_corrente(id, @s);
        SELECT @s AS saldo_corrente;
    \end{lstlisting}

    %letra C
    \item A criação da query:
    \begin{lstlisting}[language=SQL]
        DELIMITER $$

        CREATE PROCEDURE saldo_total(IN  in_id INT, OUT out_total DECIMAL(10,2))
        BEGIN
        SELECT IFNULL((SELECT SUM(saldo) FROM conta_corrente  WHERE id_conta = in_id),0) + IFNULL((SELECT SUM(saldo) FROM conta_poupanca WHERE id_conta = in_id),0)
        INTO out_total;
        END$$

        DELIMITER ;
    \end{lstlisting}

    Para retornar a procedure

    \begin{lstlisting}[language=SQL]
        SET @t = 0;
        CALL saldo_total(id, @t);
        SELECT @t AS saldo_total;
    \end{lstlisting}

    %letra D
        \item A criação da query:
    \begin{lstlisting}[language=SQL]
        DELIMITER $$

        CREATE PROCEDURE lista_usuario_email(IN in_id INT)
        BEGIN
        SELECT CONCAT(primeiro_nome, ' ', sobrenome, ' - ', email) AS `Nome / e-mail`
        FROM usuarios
        WHERE id_conta = in_id;
        END$$

        DELIMITER ;
    \end{lstlisting}

    Para retornar a procedure

    \begin{lstlisting}[language=SQL]
        CALL lista_usuario_email(id);
    \end{lstlisting}

\end{enumerate}

\section*{Exercício 7}
\paragraph{}Os adicionar os requisitos descritos a query dada pelo exercício, ela fica da seguinte maneira:

    \begin{lstlisting}[language=SQL]
        DELIMITER $$

        CREATE PROCEDURE transfiraCorrenteParaPoupanca(IN id INT, IN valor DECIMAL(10,2))
        BEGIN
        DECLARE saldo_atual DECIMAL(10,2);

        START TRANSACTION;

        SELECT saldo INTO saldo_atual
        FROM conta_corrente
        WHERE id_conta = id;

        SELECT 'Antes da transferencia:' AS info;
        SELECT saldo AS saldo_corrente FROM conta_corrente WHERE id_conta = id;
        SELECT saldo AS saldo_poupanca FROM conta_poupanca WHERE id_conta = id;

        IF valor < 0 THEN
            SIGNAL SQLSTATE '45000' SET MESSAGE_TEXT = 'Valor invalido: nao pode ser negativo';
        ELSEIF saldo_atual < 0 THEN
            SIGNAL SQLSTATE '45000' SET MESSAGE_TEXT = 'Saldo da conta corrente esta negativo';
        ELSEIF valor > saldo_atual THEN
            SIGNAL SQLSTATE '45000' SET MESSAGE_TEXT = 'Valor maior que o saldo disponivel';
        END IF;

        UPDATE conta_corrente 
        SET saldo = saldo - valor
        WHERE id_conta = id;

        UPDATE conta_poupanca
        SET saldo = saldo + valor
        WHERE id_conta = id;

        COMMIT;

        SELECT 'Depois da transferencia:' AS info;
        SELECT saldo AS saldo_corrente FROM conta_corrente WHERE id_conta = id;
        SELECT saldo AS saldo_poupanca FROM conta_poupanca WHERE id_conta = id;

        END$$

        DELIMITER ;
    \end{lstlisting}

    Para as validações, ao invés de utilizar Views, eu considerei o SIGNAL SQLSTATE '45000' uma opção melhor, pelo fato de ser mais simples do que criar uma view, já terminando a execução da procedure e, consequentemente, terminando a execução da transação, de modo que nenhuma alteração no banco é feita, além de imprimir o erro descrito na query. Ao usar o SIGNAL SQLSTATE também melhora a legibilidade e a manutenibilidade do código, considerando que todo o código está em um lugar só.

\end{document}